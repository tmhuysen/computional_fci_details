\chapter{Main}
\section{Configuration Interaction}

\begin{equation}\label{eq:ham}
      \hat{\mathcal{H}}_\text{elec} = \sum_{pq}^K h_{pq} \hat{E}_{pq} + \frac{1}{2} \sum_{pqrs}^K g_{pqrs} \hat{e}_{pqrs}
  \end{equation}

  With \textbf{h}, \textbf{g} the one and two electron integrals respectively and:
  \begin{align}
      \hat{E}_{pq} & = \hat{a}^{\beta \dagger}_p \hat{a}^{\beta}_q + \hat{a}^{\alpha \dagger}_p \hat{a}^{\alpha}_q \\
      \hat{e}_{pqrs} & = \hat{E}_{pq}  \hat{E}_{rs} - \delta_{qr} \hat{E}_{ps}
  \end{align}

  Configuration interaction problems can be solved by minimizing the expectation value of the Hamiltonian, under the constraint that the wavefunction remains normalized. This can done using the method of undetermined Langrange multipliers.

\begin{equation}\label{eq:ham}
      \mathcal{L}(\Psi_{\text{CI}}, E) = \bra{\Psi_{\text{CI}}} \hat{\mathcal{H}}_\text{elec} \ket{\Psi_{\text{CI}}} - E (\braket{\Psi_{\text{CI}}}{\Psi_{\text{CI}}} - 1)
  \end{equation}
The wave function contains all Slater Determinants $\ket{\Phi_i}$ expanded in the FCI Fock space:
\begin{equation}
  \ket{\Psi_{\text{CI}}} = \sum_i c_i \ket{\Phi_i}
\end{equation}
Evaluations of the Hamiltonian in this expansion are written as:
\begin{equation}
  \bra{\Phi_i} \hat{\mathcal{H}}_\text{elec} \ket{\Phi_j} = \text{H}_{ij}
\end{equation}
Allowing us to rewrite the Langrange equation as:

\begin{equation}
      \mathcal{L}(\{ c_a \}, E) = \sum_{ij}  c_i c_j  H_{ij} - E (\sum_i c^{2}_{i} - 1)
\end{equation}
Following the Lagrange method:
\begin{equation}
        \frac{\partial \mathcal{L}}{\partial c_i} =  \frac{\partial \mathcal{L}}{\partial E} = 0
\end{equation}
We find $\frac{\partial \mathcal{L}}{\partial c_i}$ :
\begin{equation}
    \sum_{j} c_j (H_{ij} + H_{ji}^*) = 2 E c_i
\end{equation}
As the Hamiltonian operator is Hermitian:
\begin{equation}
  \sum_{j} c_j H_{ij} = E c_i
\end{equation}
  This results in an eigenvalue problem:
\begin{equation}
      \textbf{H} \textbf{C} = \textbf{E} \textbf{C}
\end{equation}
Where \textbf{C} is the matrix of all eigenvectors which are normalised given  $\frac{\partial \mathcal{L}}{\partial E} = 0$.

\section{Constrained Configuration Interaction}
One can add additional contraints to the minimization:

\begin{equation}
      \mathcal{L}(\Psi_{\text{CI}}, E, \mu) = \bra{\Psi_\text{CI}} \hat{\mathcal{H}}_\text{elec} \ket{\Psi_{\text{CI}}} - E (\braket{\Psi_{\text{CI}}}{\Psi_{\text{CI}}} - 1) - \mu (\bra{\Psi_\text{CI}}\hat{O}\ket{\Psi_\text{CI}} - X)
  \end{equation}

  \begin{equation}
        \mathcal{L}(\{ c_a \}, E, \mu) = \sum_{ij}  c_i c_j  H_{ij} - E (\sum_i c^{2}_{i} - 1) - \mu (\sum_{ij} c_i c_j O_{ij} - X)
    \end{equation}
Where $\hat{O}$ is a given electron operator (of any order) and $X$ a value to which we constrain the expectation value of $\hat{O}$.
We are now minimizing the constrained Hamiltonian.
\begin{equation}\label{eq:constrained_ham}
  \hat{H}^c = \hat{\mathcal{H}}_\text{elec}  - \mu (\hat{O} - X)
\end{equation}
or
\begin{equation}\label{eq:constrained_ham_X}
  \hat{H}^c = \hat{\mathcal{H}}_\text{elec}  - \mu (\hat{O}^X)
\end{equation}

\begin{equation}
      \frac{\partial \mathcal{L}}{\partial c_i} =  \frac{\partial \mathcal{L}}{\partial E} = \frac{\partial \mathcal{L}}{\partial \mu} = 0
\end{equation}


%%% EDITE
We find an eigenvalue problem $\frac{\partial \mathcal{L}}{\partial c_i} = 0$:
\begin{equation}\label{eq:perturbed_expanded}
  2 \sum_{j} c_j H_{ij} - \mu \sum_{j} c_j (O_{ij} + O_{ji}^*)  = 2 \mathcal{E} c_i
\end{equation}

\begin{equation}\label{eq:perturbed_matrix}
      (\textbf{H} - \frac{\mu}{2} (\textbf{O} + \textbf{O}^\dagger)) \textbf{C} = \mathbfcal{E}  \textbf{C}
\end{equation}
Where \textbf{C} is the matrix of all eigenvectors which are normalised given  $\frac{\partial \mathcal{L}}{\partial E} = 0$.

The condidtion:

\begin{equation}
  \frac{\partial \mathcal{L}}{\partial \mu} = \bra{\Psi} \hat{O} \ket{\Psi} - X = 0
\end{equation}
Appears to difficult to sastifiy analitically. If we ignore this condition we find the equation for minimizing the perturbed Hamiltonian:

\begin{equation}\label{eq:perturbed_ham}
    \hat{{H}}^{p} =  \hat{\mathcal{H}}_\text{elec} - \mu \hat{O}
\end{equation}
So that we can rewrite equation (\ref{eq:perturbed_temp}) as:
\begin{equation}
      \textbf{H}^{p} \textbf{C} = \mathbfcal{E}  \textbf{C}
\end{equation}


\subsection{Solutions for Constrained Configuration Interaction} \label{solutions}

  Zeiss et. al. \cite{zeiss1983constrainedci} examines two iterative procedure to find a $\mu$ for which a given operator $\hat{O}^X$ has a vanishing expectation value for the groundstate of the constrained Hamiltonian, of which the the secant-parameterization \cite{zeiss1983constrained} was the most computationally feasible for CI-methods. The method starts of with two guesses for $\mu$ which are plugged into the Hamiltonian (which can be considered perturbed as the constraint is not met at the start). Based on the retrieved expectation value for $\hat{O}^X$ the multiplier is modified and the Hamiltonian is solved again until convergence requirements are met.

  Let simply choose a multiplier $\mu$. The eigen vector spectrum of the Hamiltonian for given system is characterized by $\mu$ and and the given operator $\hat{O}$. The presence of $X$ does only affect the eigenvalues as it is a scalar. So for a given $\mu$ the perturbed Hamiltonian (\ref{eq:perturbed_ham}) and the constrained Hamiltonian (\ref{eq:constrained_ham_X}) have an identical set of eigenvectors.

  The relation between $\hat{H}^{p}$ and $\hat{H}^c$ allows us to simply choose a $\mu$ and calculate the $X$ and arrive at the same solution $\hat{H}^c$ would have for the given $X$. This is advantagous if one wants to cover a large set of constraints. One can start from a grid of multipliers and interpolate the results to generate set of new mulipliers which should result in a set of requested expectation values for the operator $\hat{O}$. See figure \ref{fig:alg1}. Interpolation of data was done with the intepolate module of "SciPy" \cite{scipy}.

  \begin{figure}[h!]
    \includegraphics[width=\linewidth]{figures/interpolate.pdf}
    \caption{Iterative interpolation procedure}
    \label{fig:alg1}
  \end{figure}

\section{Mulliken Operator}

  To derive the Mulliken operator in second quantization we first define:
  \begin{align}
    [ \hat{a}_p^\dagger, \hat{a}_q ]_+ &= \delta_{pq} \\
      [ \hat{b}_\lambda^\dagger, \hat{b}_\nu ]_+ &= S_{\nu \lambda} \\
  \hat{b}_{\lambda}^\dagger &= \sum_p (U S^{1/2})_{p \lambda} \hat{a}_p^\dagger \\
    \hat{b}_\lambda &= \sum_p ( U S^{1/2})^*_{p \lambda} \hat{a}_p \\
  \widetilde{b}_\lambda &= \sum_\nu S_{\lambda \nu}^{-1} \hat{b}_\nu \\
    \widetilde{b}^\dagger_\lambda &= \sum_\nu S_{\nu \lambda}^{-1} \hat{b}^\dagger_\nu \\
  [\widetilde{b}_\nu,\widetilde{b}^\dagger_\lambda ] &= S^{-1}_{\nu \lambda}
  \end{align}
  In which \textbf{S} is the overlap matrix of the basis functions.
  %
  The number operator for a given basis function in the (non-orthgonal) atomic basis can be written as\cite{surjan2012second}:
  \begin{equation}
   \hat{n}_\lambda = \hat{b}_{\lambda}^\dagger \widetilde{b}_\lambda
  \end{equation}
  We can re-write this in terms of the orthongal operators $\hat{a}_p^\dagger, \hat{a}_q$.
  \begin{align}
  \hat{n}_\lambda &= \hat{b}_{\lambda}^\dagger \widetilde{b}_\lambda \\
  \hat{n}_\lambda &= \sum_\nu S_{\lambda \nu}^{-1} \hat{b}_\lambda^\dagger \hat{b}_\nu \\
  \hat{n}_\lambda &= \sum_\nu \sum_p (U S^{1/2})_{p \lambda} S_{\lambda \nu}^{-1} \hat{a}_p^\dagger \hat{b}_\nu \\
  \hat{n}_\lambda &= \sum_\nu \sum_p \sum_q  (U S^{1/2})_{p \lambda} S_{\lambda \nu}^{-1} ( U^* S^{1/2})_{q \nu}  \hat{a}_p^\dagger \hat{a}_q \\
  \hat{n}_\lambda &= \sum_p \sum_q  (U S^{1/2})_{p \lambda} (S^{-1/2} U^\dagger)_{\lambda q} \hat{a}_p^\dagger \hat{a}_q \\
  \hat{n}_\lambda &= \sum_p \sum_q  (U S^{1/2} P^\lambda S^{-1/2} U^\dagger)_{pq} \hat{a}_p^\dagger \hat{a}_q
  \end{align}
  $\textbf{P}^\lambda$ is a zero matrix with one diagonal element with the index corresponding to the atomic orbital $\lambda$ is set to one.
  We are working with K spatial orbitals in restricted space hence we can write:
  \begin{equation}
     \hat{n}_\lambda = \sum^K_{pq} (U S^{1/2} P^\lambda S^{-1/2} U^\dagger)_{pq} \hat{E}_{pq}
  \end{equation}
  %
  Taking $\textbf{T}= \textbf{U} \textbf{S}^{-1/2}$
  \begin{equation}
     \hat{n}_\lambda =\sum^K_{pq} (T S P^\lambda T^\dagger)_{pq} \hat{E}_{pq}
  \end{equation}


  We can extend this operator for all atomic orbitals centered on atom A:
  \begin{equation}
     \hat{n}_\text{A} = \sum^K_{pq} (T {\color{black}{S}} {\color{black}{P}}^{\color{black}{A}} T^\dagger)_{pq} \hat{E}_{pq}
  \end{equation}
  Now we have $\textbf{P}_A$ a partitioning matrix, which is a diagonal matrix where the diagonal indices corresponding to atomic orbitals centered on A are set to one. E.g. take a diatomic molecule with atom A and B with each 3 atomic basis functions:


  \begin{table}[H]
  \centering
  \captionsetup{justification=centering,margin=2cm}

  \begin{tabular}{|c||c|c|c|c|c|c|c|c|c|}
  \hline
   & $\lambda^{A}_1$ & $\lambda^{A}_2$  &  $\lambda^{A}_3$ &  $\lambda^{B}_1$ & $\lambda^{B}_2$  &  $\lambda^{B}_3$ \\ \hline \hline
  $\lambda^{A}_1$ & 1 & \textcolor{lightgray}{0} & \textcolor{lightgray}{0} & \textcolor{lightgray}{0} & \textcolor{lightgray}{0} & \textcolor{lightgray}{0} \\ \hline
  $\lambda^{A}_2$ & \textcolor{lightgray}{0} & 1 & \textcolor{lightgray}{0} & \textcolor{lightgray}{0} & \textcolor{lightgray}{0} & \textcolor{lightgray}{0} \\ \hline
  $\lambda^{A}_3$ & \textcolor{lightgray}{0} & \textcolor{lightgray}{0} & 1 & \textcolor{lightgray}{0} & \textcolor{lightgray}{0} & \textcolor{lightgray}{0} \\ \hline
  $\lambda^{B}_1$ & \textcolor{lightgray}{0} & \textcolor{lightgray}{0} & \textcolor{lightgray}{0} & \textcolor{lightgray}{0} & \textcolor{lightgray}{0} & \textcolor{lightgray}{0} \\ \hline
  $\lambda^{B}_2$ & \textcolor{lightgray}{0} & \textcolor{lightgray}{0} & \textcolor{lightgray}{0} & \textcolor{lightgray}{0} & \textcolor{lightgray}{0} & \textcolor{lightgray}{0} \\ \hline
  $\lambda^{B}_3$ & \textcolor{lightgray}{0} & \textcolor{lightgray}{0} & \textcolor{lightgray}{0} & \textcolor{lightgray}{0} & \textcolor{lightgray}{0} & \textcolor{lightgray}{0} \\ \hline

  \end{tabular}
  \caption {Partitioning matrix for atomic basis functions centered on atom A}

  \end{table}
  It can be noted this operator is not Hermitian. For ease of use we define the Hermitian Mulliken operator:
  \begin{equation}
    \hat{m}^A = \frac{\hat{n}^A + (\hat{n}^A)^\dagger}{2}
  \end{equation}
  This does not affect the Mulliken population analysis of a given real normalised vector $\ket{\Psi}$:
  \begin{align}
   \bra{\Psi} (\hat{n}^{A})^\dagger \ket{\Psi} & =   \bra{\Psi} \hat{n}^{A} \ket{\Psi} \\
    \bra{\Psi} \hat{m}^{A} \ket{\Psi} & =  \bra{\Psi} \hat{n}^{A} \ket{\Psi}
  \end{align}
  Defining:
  \begin{equation}
  \textbf{m}^{A} = \frac{\textbf{T}^{\dagger}  \textbf{P}^A  \textbf{S} \textbf{T} + \textbf{T}^{\dagger}  \textbf{S}  \textbf{P}^A \textbf{T}}{2}
  \end{equation}
  We can write:
  \begin{equation}
  \hat{m}^{A} = \sum^K_{pq} m^A_{pq}  \hat{E}_{pq}
  \end{equation}
  
  %
  %  Spin Constraints
  %

  \subsection{Spin Contraints}
  The $z$-spin of an atomic fragment $A$ will be defined by the atomic $z$-component operator $\hat{s}^{z}_A$:
  \begin{equation}
   \hat{s}^{z}_A = \sum_p^A \hat{S}^{z}_p = \sum_p^A \frac{1}{2} (\hat{b}_{p \alpha}^\dagger \widetilde{b}_{p \alpha} - \hat{b}_{p \beta}^\dagger \widetilde{b}_{p \beta})
  \end{equation}
  This can be transformed (see Mulliken)
  \begin{equation}
   \hat{s}^{z}_A = \sum_{pq} (T S P^A T^\dagger)_{pq} \frac{1}{2} (\hat{a}_{p \alpha}^\dagger \hat{a}_{q \alpha} - \hat{a}_{p \beta}^\dagger \hat{a}_{q \beta})
  \end{equation}

  This requires an unrestricted basis. But note that only the one electron integrals are modified and the LCAO's are identical for beta and alpha. 

  It can be noted this operator is not Hermitian. For ease of use we define the Hermitian atomic $z$-spin operator:  

  \begin{equation}
    \hat{S}^{z}_A = \sum_{pq}  \frac{(T S P^A T^\dagger + T P^A S T^\dagger)_{pq}}{4}  (\hat{a}_{p \alpha}^\dagger \hat{a}_{q \alpha} - \hat{a}_{p \beta}^\dagger \hat{a}_{q \beta})
  \end{equation}

\section{Population Constrained}
  We can now pertrub the Hamiltonian for a single atom (defined by its atomic basis functions) we can write the singular constrained Hamiltonian:
  \begin{equation}\label{eq:constrained_atom_A}
    \hat{{H}}^{A} = \hat{\mathcal{H}}_\text{elec} - \mu (\hat{m}^{A} - N^A)
  \end{equation}
  Or use the perturbed Hamiltonian:
  \begin{equation}\label{eq:perturbed_atom_A}
    \hat{{H}}^{A} = \hat{\mathcal{H}}_\text{elec} - \mu (\hat{m}^{A})
  \end{equation}

\section{Considerations \& Issues}

 Results for $\hat{H}^c$ can be seen in figure \ref{fig:res1} and \ref{fig:res2}.

 \begin{figure}[h!]
   \includegraphics[width=\linewidth]{figures/ppe}
   \caption{Expectation values for the Hamiltonian for the $NO^+$ molecule in function of the population of N for a range of constraints on the population of N at an different intramolecular distances, followed by the value of the chemical potential in function of that same population}
   \label{fig:res1}
 \end{figure}


  \begin{figure}[h!]
    \includegraphics[width=\linewidth]{figures/chemical}
    \caption{Expectation values for the Hamiltonian for the $NO^+$ molecule in function of the chemical potential on N for a range of constraints on the population of N at an different intramolecular distances}
    \label{fig:res2}
  \end{figure}

  One can call these the eigenvalues of $\hat{H}^c$ or the expectation values for $\hat{H}$ obtained from an eigenvector of $\hat{H}^c$. This is logical for we have a $\mu$ for which the operator $\hat{O}^X$ has a vanishing expectation value, the constraint is (in general) not sastisfied for the complete set of eigenvectors as it would be extremely unlikely for the expectation value of $\hat{O}$ to be identical for all eigenvectors. And therefore only the groundstate eigenvalue will correspond to the expectation value for $\hat{H}$ for its eigenvector. A "perfect" or complete constraint should be vanishing for all eigenvectors. So that the eigenvalues of the complete constrained Hamiltonian $\hat{H}^{cc}$ and the expectation value of $\hat{H}$ are identical, however this requires as many values for $X$ as the dimension of the Fock space!

  \begin{equation}
    \hat{H}^{cc} = \hat{\mathcal{H}}_\text{elec}  - \mu (\hat{O} - \textbf{X}^{cc})
  \end{equation}

  It is clear that $\mu$ and these expectation values of the constraining operator are linked, meaning that it is not feasible to search for the vanishing operator constaint for more than one eigenvector.

  Is now interesting to note that while $\hat{H}^{p}$ and $\hat{H}^c$ and $\hat{H}^{cc}$ share an identical set of eigenvectors, and while the order from groundstate to highest excited state for a given $\mu$ are identical for $\hat{H}^{p}$ and $\hat{H}^c$ (the constrained Hamiltonian has the same shift for all eigenvalues, they are not universally correct for their associated expectation value of the constraining operator for each eigenvector), this is not necessarily the case for $\hat{H}^{cc}$ as for every eigenvector the expectation value of $\hat{O}^{X_i}$ is vanishing, as if each eigenvector has its own operator. In otherwords the eigenvectors of  $\hat{H}^{p}$ and $\hat{H}^c$ are not necessarily ordered according to the expectation values obtained for the electronic Hamiltonian (equation \ref{eq:ham}) while those of $\hat{H}^{cc}$ are, as the set eigenvalues of $\hat{H}^{cc}$ and the expecation values obtained for $\hat{H}_{\text{elec}}$ are identical.

  \begin{align}
    \ket{\Psi^{cc}_0} & \neq \ket{\Psi^{c}_0}  \\
    E^{cc}_0 &< E^{c}_0 \\
    X^{cc}_0 &\neq X^{c}_0 \\
    \ket{\Psi^{cc}_0} & = \ket{\Psi^{c}_i}  \\
    E^{c}_i &> E^{c}_0 \\
    E^{c}_0 - E^{c}_i &> - \mu (X^{c}_i - X^{c}_0)
  \end{align}
  With $X^{c}_0$ the value of the requested groundstate constraint (and thus vanishing) and $X^{c}_i$ a non-vanishing expectation value obtained for an excited state.

\begin{figure}[h!]
  \includegraphics[width=\linewidth]{figures/edit-arrow-ep.pdf}
  \caption{Eigenvalues for the complete constrained Hamiltonian for the first 10 states of $NO^+$ in function of the population of N for a range of constraints on the population of N at an intermolecular distance of 2.3 bohr}
  \label{fig:complete}
\end{figure}

\begin{figure}[h!]
  \includegraphics[width=\linewidth]{figures/2-3-complete-0th-E-P.pdf}
  \caption{Eigenvalues for the complete constrained Hamiltonian for the ground state of $NO^+$ in function of the population of N for a range of constraints on the population of N at an intermolecular distance of 2.3 bohr}
  \label{fig:complete}
\end{figure}

\begin{figure}[h!]
  \includegraphics[width=\linewidth]{figures/2-3-singular-E-P.pdf}
  \caption{Lowest found eigenvalues for the complete constrained Hamiltonian for target populations of N of NO+ in function of the population of N for a range of constraints on the population of N at an intermolecular distance of 2.3 bohr}
  \label{fig:singular}
\end{figure}

\begin{figure}[h!]
  \includegraphics[width=\linewidth]{figures/edit-arrow-pc.pdf}
  \caption{Populations for the complete constrained Hamiltonian for the first state of $NO^+$ in function of the chemical potentian $\mu$ of N for a range of constraints on the population of N at an intermolecular distance of 2.3 bohr}
  \label{fig:singular}
\end{figure}


\begin{figure}[h!]
  \includegraphics[width=\linewidth]{figures/Potential_Energy2-3dot.pdf}
  \caption{Eigenvalues for the complete constrained Hamiltonian for the first 10 states of $NO^+$ in function of the chemical potentian $\mu$ of N for a range of constraints on the population of N at an intermolecular distance of 2.3 bohr}
  \label{fig:singular}
\end{figure}

So this means that for a given $\mu$ we are not garatueed to retrieve the lowest expectation value for that $\mu$.
This is computationally demonstrated by calculating (and only partially displaying) the eigenspectrum from $\hat{H}^{cc}$ as seen in figure \ref{fig:complete} and \ref{fig:complete-0th}.
If instead we display the expectation value aquired from the groundstate of $\hat{H}^c$ or $\hat{H}^p$ we arrive at figure \ref{fig:singular}. One can see that  this curve coicides with that of the lowest curve by population for the spectrum of $\hat{H}^{cc}$ on figure \ref{fig:complete}, infact when we solve $\hat{H}^{cc}$ dense, then and look for the lowest energetic expectation value for a grid of population expecation values then we indeed arrive at an identical curve. It thus appears that we still find the best possible energy for a given expectation value of the constrained operator. However this does not prove that there does not exists a different constraint : a different $\mu$ with the same $N^A$ but a lower expectation value.
\subsubsection{Questions to myself}

\begin{itemize}
  \item Can I prove my solutions are the best for a given population?
  \item Should I disregard the excited solutions completely as they do not have the population expectation values that were targeted.
  \item What is the meaning of $\mu$ the "chemical potential"? Is it the derivative $\frac{\delta E}{\delta N}$ if so, then $\mu$ for with respect to excited states of $H^{cc}$ does not seem to adhere to this idea.
  \item Could this ordering of eigenvectors cause problems? Can I propose a valid system and target constraint for which the groundstates of $\hat{H}^c$ will not have the desired outcomes and instead $\hat{H}^{cc}$ should be fully explored?
  \item Maybe the most meaningfull ordering is that of $H^c$ (or $H^p$)
 \end{itemize}


\subsection{Atomic Decomposition}

  A way to decompose the contributions of a diatomic molecule to the total energy was defined \cite{mayer2005atomic}. The self interacting atomic Hamiltonian for atom A is defined:

  \begin{equation}
    \hat{H}_{AA} = \sum_{pq}^K (P^A (\mathcal{T}^{\text{AO}} + V_{A}^{\text{AO}}) P^A)_{pq} \widetilde{E}_{pq} +  \frac{1}{2} \sum_{ij} \sum_{pqrs}^K P^{A}_{pi} P^{A}_{rj} g^{\text{AO}}_{iqjs} \widetilde{e}_{pqrs}
  \end{equation}

  In which $\mathcal{T}^{\text{AO}}$ are the kinetic operator evaluations for the atomic orbitals and $V_A^{\text{AO}}$ nuclear electronic attractions with respect to the atom A such that $\hat{H} = T + \sum_A V_A$. Additionally:
  \begin{align}
      \widetilde{e}_{pqrs} &= \sum_\sigma \widetilde{b}^{\sigma \dagger}_p  \widetilde{E}_{rs}  \widetilde{b}^{\sigma}_q \\
       \widetilde{E}_{pq} &= \widetilde{b}^{\beta \dagger}_p  \widetilde{b}^\beta_q + \widetilde{b}^{\alpha \dagger}_p  \widetilde{b}^\alpha_q
  \end{align}

  For a diatomic molecule we can then calculate the interaction Hamiltonian $\hat{H}_{AB}$ and atomic Hamiltoninian $\hat{H}_{A}$:
  \begin{align}
    \hat{H}_{AB} &= \hat{H} -   \hat{H}_{AA} -  \hat{H}_{BB} \\
     &= \sum_{pq}^K (P^A \mathcal{T}^{\text{AO}} P^B + P^B \mathcal{T}^{\text{AO}} P^A)_{pq} \widetilde{E}_{pq} \\
     &+\sum_{pq}^K (P^B V_{A}^{\text{AO}} + V_{A}^{\text{AO}} P^B - P^B V_A P^B)_{pq} + (P^A V_{B}^{\text{AO}} + V_{B}^{\text{AO}} P^A - P^A V_B P^A)_{pq} \widetilde{E}_{pq} \\
     &+ \frac{1}{2} \sum_{ij} \sum_{pqrs}^K (P^{A}_{pi} P^{B}_{rj} g^{\text{AO}}_{iqjs} +  P^{B}_{pi} P^{A}_{rj} g^{\text{AO}}_{iqjs}) \widetilde{e}_{pqrs} + V_{\text{nuc}}  \\
    \hat{H}_A &= \hat{H}_{AA} + \frac{1}{2}   \hat{H}_{AB}
  \end{align}

  The entities $E_{AA}$, $E_{AB}$, $E_{A}$, ... are then defined as:
  \begin{equation}
    E_{AA} = \bra{} \hat{H}_{AA} \ket{} = \sum_{pq}^K (P^A (\mathcal{T}^{\text{AO}} + V_{A}^{\text{AO}}) P^A)_{pq} D^{AO}_{pq} +  \frac{1}{2} \sum_{ij} \sum_{pqrs}^K P^{A}_{pi} P^{A}_{rj} g^{\text{AO}}_{iqjs}  d^{AO}_{pqrs}
  \end{equation}
  


\subsubsection{Spin Decomposition}

As seen in \cite{mayer2005atomic} we can apply decomposition to any one-electron operator through partition. 
For spin we define the value $S_z^A$:

\begin{equation}
  S_z^A =  \bra{} \hat{S_z}^A \ket{} = Tr(\textbf{S}^A_z (\textbf{D}_\alpha - \textbf{D}_\beta)) = Tr(\textbf{s}^A_z (\textbf{D}_\alpha - \textbf{D}_\beta))
\end{equation}


\section{Entropy}
  \subsection{Spin Orbitals}
  Let us define a wavefunction :
  \begin{equation}
    \ket{\Psi} = \sum_i c_i \ket{\textbf{i}}
  \end{equation}
  With ONVs:
  \begin{align}
      \ket{\textbf{i}} &= \ket{i_1, ..., i_M} ; i_P = 0,1 \\
      &= \prod^{N}_{P} a^{\dagger}_P \ket{vac} ; \forall i_P = 1
  \end{align}
  In which $M$ is equal to the amount of orbitals and $N$ is equal to the amount electrons.
  Given two arbitraily defined fragments A and B, we can attribute indices in the ONV to them as:
  \begin{align}
  \ket{\textbf{i}} &= \ket{i_{A_1},...,i_{A_m},i_{B_1},...,i_{B_m}} \\
   &= \prod^{A_n}_{P} a^{\dagger}_P \ket{vac} \otimes \prod^{B_n}_{Q}  a^{\dagger}_Q \ket{vac} \\
   &= \ket{\textbf{i}^A} \otimes \ket{\textbf{i}^B}
  \end{align}
  in which $A_m + B_m = M$. and $A_n + B_n = N$ for which we find that:
  \begin{align}
    A_n &\geq 0 \\
    A_n &\geq N - B_n \\
    B_n &\geq 0 \\
    B_n &\geq N - A_n
  \end{align}
  It should be noted that $A_m$ and $B_m$ are fixed at definition of the fragments, while all integer combinations for $A_n$ and $B_n$ are still possible while the conditions above are met. Therefor for a given $A_n$ we have $\binom{A_m}{A_n} \binom{M-A_m}{N-A_n}$ or $\binom{A_m}{A_n} \binom{B_m}{B_n}$ combinations. Let us define $F^B(B_n) = \binom{B_m}{B_n}$ and $F^A(A_n)=\binom{A_m}{A_n}$.
  Let us define the density matrix ${\rho}$:
  \begin{equation}
  \rho_{ij} = \bra{i} \ket{\Psi} \bra{\Psi} \ket{j}
  \end{equation}
  We can define $\rho^A$ as:

  \begin{equation}
    \rho_{ij}^A = \bra{i^A} \otimes \sum_{B_n=0}^{B_m} \sum_{k}^{F^B(B_n)} \bra{k^B} \ket{\Psi} \bra{\Psi} \ket{k^B}  \otimes \ket{j^A}
  \end{equation}

We sum over all possible $F^B(B_n)$ we iterate over all possible sub-Fock spaces of B. This results in an onv from A and B combined not being part of the original Fock space, these combinations result in a zero. As $\bra{k^B}$ is a fixed sum for both the bra and ket for the ONVs on A this means that only ONVs from A with the same spin and electron number result in non-zero elements resulting a blocked DM.

  The entropy is then defined as:
  \begin{equation}
    S = - \sum_i \lambda_i log_2 (\lambda_i)
  \end{equation}
  with $\lambda$ the eigenvalues of $\rho^A$
\end{document}
