\documentclass[12p]{article}
\usepackage[english]{babel}
%\usepackage{microtype}
\usepackage[left=3cm, right=3cm]{geometry}		                                % Margins left and right
\usepackage{hyperref}                                                           % Clickable table of contents in PDFs
\usepackage{datetime}                                                           % Currenttime
\usepackage{graphicx}                                                           % Import .pdf-files as figure
\usepackage{subcaption}                                                         % Subfigure
\usepackage{float}
\usepackage{listings}
\usepackage{physics}                                                            % Physics typesetting: bra-ket
\usepackage{amsfonts}                                                           % mathfrak
\usepackage{mathtools}                                                          % dcases, DeclarePairedDelimiter
\usepackage{amsmath}
\usepackage{xcolor}
\definecolor{light-gray}{gray}{0.95}
\newcommand{\code}[1]{\colorbox{light-gray}{\texttt{#1}}}
\usepackage{tikz}
\usepackage[]{algorithm2e}
\usepackage{wrapfig}
\usepackage{standalone}


\title{FCI: Computational Details}
\author{}

\begin{document}

\maketitle

\section{Two-Electron Off-Diagonal Address Calculation}

Given a Fock space.
\\
Let us call $\ket{I_{\alpha}}$ the ONV with address I.
We have arrived at following expression for the Hamiltonian:

      \begin{equation}\label{eq:ham}
          \hat{\mathcal{H}}_\text{elec} = \sum_{pq}^K k_{pq} \hat{E}_{pq} + \frac{1}{2} \sum_{pqrs}^K g_{pqrs} \hat{E}_{pq} \hat{E}_{rs}
      \end{equation}
We will focus only on the two electron operators $ \sum_{pqrs}^K g_{pqrs} \hat{a}^{\dagger}_{p} \hat{a}_{q}  \hat{a}^{\dagger}_{r} \hat{a}_{s}$. And only focus on the $\alpha$ electrons. We will therefore ignore the $\alpha$ subscript.
We require to consider operator indices for which a given ONV I does not vanish:
\begin{equation} \label{eq:address}
  \bra{I} \hat{a}^{\dagger}_{p} \hat{a}_{q} \hat{a}^{\dagger}_{r} \hat{a}_{s} = \bra{J}
\end{equation}
In which $J$ is an address larger than $I$. Reason for this is, in the event that $\ket{I}$ can be transformed in $\ket{J}$. $\ket{J}$ can also be transformed back to $\ket{I}$ yielding the same two-electron term (hermitian two-electron operators).

\begin{equation}
  (\bra{I} E_{pq} E_{rs} \ket{J})^\dagger = \bra{J} E_{sr} E_{qp} \ket{I}
\end{equation}

It is important to note, that I will base further explanations from the perspective of equation (\ref{eq:address}) where $\hat{a}^{\dagger}_{p}$ annihilates on $\bra{I}$.

\subsection{Minimal operator iterations}
For the address to be larger at all times, the highest index of a creation should always be higher than the highest index of an annihilation. This is easily verified by the fact that we represent our ONVs in binary and that the addressing is based on the ordering of its integer value. Given the relation of numeric value for each index of an integer represented in binary is quadratic, the integer value of a set index is always larger than any combination of previously set indices:
\begin{equation}
  2^N - 1 = \sum^{N-1}_{i=0} 2^i
\end{equation}
Additionally we can state that the first annihiltion can always have a smaller index than the second annihiltion without skipping over any address, the same is true for the creation operators.

Regardless is which way they are executed (if they are all different indices), the address will be the same. However the order of execution can alter the expression (sign wise) and will be accompanied by a different two-electron term. Given an ONV $\bra{I}$ for which $\bra{I} \hat{a}^\dagger_p \hat{a}_q \hat{a}^\dagger_r \hat{a}_s \neq 0 $ we find that:
\begin{align}
  \bra{I} \hat{a}^\dagger_p \hat{a}_q \hat{a}^\dagger_r \hat{a}_s & = \bra{I} (\hat{a}^\dagger_p \hat{a}_s \delta_{rq} - \hat{a}^\dagger_p \hat{a}^\dagger_r \hat{a}_q \hat{a}_s) \nonumber \\
  & = \bra{I} (\hat{a}^\dagger_p  \hat{a}_s \delta_{rq} + \hat{a}^\dagger_r \hat{a}^\dagger_p \hat{a}_q \hat{a}_s) \nonumber \\
  & = \bra{I} (\hat{a}^\dagger_p  \hat{a}_s \delta_{rq} + \hat{a}^\dagger_r \hat{a}_s \delta_{pq} - \hat{a}^\dagger_r \hat{a}_q \hat{a}^\dagger_p \hat{a}_s) \label{eq:equals} \\
  & \text{IF p,q,r,s $\neq$} \nonumber \\
  & = - \bra{I}(\hat{a}^\dagger_r \hat{a}_q \hat{a}^\dagger_p \hat{a}_s )\\
  & = \bra{I} (\hat{a}^\dagger_r \hat{a}_s \hat{a}^\dagger_p \hat{a}_q ) \\
  & = - \bra{I} (\hat{a}^\dagger_p \hat{a}_s \hat{a}^\dagger_r \hat{a}_q)
\end{align}
So for the \textit{p,q,r,s $\neq$} case, we can enforce : $p<r$, $q<s$ for symmetries and anti-symmetries, and $s > r$ upper diagonal to not generate redundant addresses. This leaves us with limited combinations:

\begin{enumerate}
  \item $p > q$ ($s > r > p$)
  \item $p < q$
  \begin{itemize}
    \item $r > q$ ($s > r$)
    \item $q > r$ ($s > q$)
  \end{itemize}
\end{enumerate}
For inplace annihila-crea- and crea-annihila-tions, the rules are slightly different, because creation annihilation operators with the same index cancel each other out. Therefore the non-annihiltion bound creation index has to be larger than the non-creation bound annihiltion index to produce larger addresses (the rules for one-electron evaluation).

\begin{enumerate}
  \item $p = q$, $s > r$
  \item $q = r$, $s > p$
\end{enumerate}
These have some implication for symmetry and anti-symmetry equations such as equation (\ref{eq:equals}) as $\delta$ is not always zero:


\begin{align}
   \bra{I}  \hat{a}^\dagger_p \hat{a}_q \hat{a}^\dagger_r \hat{a}_s & =   \bra{I} (\hat{a}^\dagger_p \hat{a}_s \delta_{rq} + \hat{a}^\dagger_r  \hat{a}_s \delta_{pq} - \hat{a}^\dagger_r \hat{a}_q \hat{a}^\dagger_p \hat{a}_s) \\
  & =   \bra{I} (\hat{a}^\dagger_p \hat{a}_s \delta_{rq} + \hat{a}^\dagger_r \hat{a}_s \delta_{pq} - \hat{a}^\dagger_r \hat{a}_q  \delta_{sp} - \hat{a}^\dagger_r \hat{a}_s \delta_{pq} + \hat{a}^\dagger_r \hat{a}_s \hat{a}^\dagger_p \hat{a}_q) \\
  & =   \bra{I} (\hat{a}^\dagger_p \hat{a}_q \delta_{rs} + \hat{a}^\dagger_p \hat{a}_s \delta_{qr} - \hat{a}^\dagger_p \hat{a}_s \hat{a}^\dagger_r \hat{a}_q)
\end{align}

\subsubsection{$p = q$} \label{p=q}
For $p = q$ we can also see that for $s = q \iff r = q$ ortherwise we would have a double creation on the same index without an annihilation on that same index, which is a vanishing operation sequence. However this does not alter the address (diagonal contribution) and is ignored in the algorithm. Hence we state that $ s \neq p, q, r $ This simplifies the equations:

\begin{align}
  \bra{I}  \hat{a}^\dagger_p \hat{a}_p \hat{a}^\dagger_r \hat{a}_s & =  \bra{I} (\hat{a}^\dagger_p \hat{a}_s \delta_{rp} + \hat{a}^\dagger_r  \hat{a}_s - \hat{a}^\dagger_r \hat{a}_p \hat{a}^\dagger_p \hat{a}_s) \\
  & =   \bra{I} (\hat{a}^\dagger_p \hat{a}_s \delta_{rp} + \hat{a}^\dagger_r \hat{a}_s \hat{a}^\dagger_p \hat{a}_p) \\
  & =   \bra{I} (\hat{a}^\dagger_p \hat{a}_s \delta_{pr} - \hat{a}^\dagger_p \hat{a}_s \hat{a}^\dagger_r \hat{a}_p)
\end{align}
We can then discriminate between $r = p$ :

\begin{align}
    \bra{I} \hat{a}^\dagger_p \hat{a}_p \hat{a}^\dagger_p \hat{a}_s & =   \bra{I} (\hat{a}^\dagger_p \hat{a}_s + \hat{a}^\dagger_p \hat{a}_s - \hat{a}^\dagger_p \hat{a}_p \hat{a}^\dagger_p \hat{a}_s ) \\
  & =   \bra{I} (\hat{a}^\dagger_p \hat{a}_s + \hat{a}^\dagger_p \hat{a}_s \hat{a}^\dagger_p \hat{a}_p) \label{eq:vanish1} \\
  & =   \bra{I} (\hat{a}^\dagger_p \hat{a}_s - \hat{a}^\dagger_p \hat{a}_s \hat{a}^\dagger_p \hat{a}_p) \label{eq:vanish2}
\end{align}
We see in equation (\ref{eq:vanish1}) and (\ref{eq:vanish2}) that last term annihilates $p$, then operator on index $s$ does strictly not create on index $p$ and index $p$ is annihilated again, thus this term vanishes:

\begin{equation}
 \bra{I}  \hat{a}^\dagger_p \hat{a}_p \hat{a}^\dagger_p \hat{a}_s = \bra{I} \hat{a}^\dagger_p \hat{a}_s
\end{equation}
For $r \neq p$:
\begin{align}
    \bra{I} \hat{a}^\dagger_p \hat{a}_p \hat{a}^\dagger_r \hat{a}_s & =   \bra{I} ( \hat{a}^\dagger_r  \hat{a}_s - \hat{a}^\dagger_r \hat{a}_p \hat{a}^\dagger_p \hat{a}_s ) \nonumber \\
  & =   \bra{I} (\hat{a}^\dagger_r  \hat{a}_s) \label{eq:rneqp}\\
  & =   \bra{I} ( \hat{a}^\dagger_r \hat{a}_s \hat{a}^\dagger_p \hat{a}_p) \\
  & = -   \bra{I} (\hat{a}^\dagger_p \hat{a}_s \hat{a}^\dagger_r \hat{a}_p)
\end{align}
Where se see that for equation (\ref{eq:rneqp}) the second term vanished, as the initial term is assumend non-vanishing.

\subsubsection{$q=r$}
We only cover $p \neq r$ as we assume tha $r$ starts unoccupied as opposed to the previous section (\ref{p=q}).
\begin{align}
  \bra{I} \hat{a}^\dagger_p \hat{a}_r \hat{a}^\dagger_r \hat{a}_s & =   \bra{I} (\hat{a}^\dagger_p \hat{a}_s - \hat{a}^\dagger_r \hat{a}_r \hat{a}^\dagger_p \hat{a}_s) \nonumber \\
  & =   \bra{I} \hat{a}^\dagger_p \hat{a}_s  \\
  & =   \bra{I} (\hat{a}^\dagger_p \hat{a}_s + \hat{a}^\dagger_r \hat{a}_s \delta_{pr} - \hat{a}^\dagger_r \hat{a}_r  \delta_{sp} - \hat{a}^\dagger_r \hat{a}_s \delta_{pr} + \hat{a}^\dagger_r \hat{a}_s \hat{a}^\dagger_p \hat{a}_r) \\
  & =   \bra{I} (\hat{a}^\dagger_p \hat{a}_r \delta_{rs} + \hat{a}^\dagger_p \hat{a}_s - \hat{a}^\dagger_p \hat{a}_s \hat{a}^\dagger_r \hat{a}_r)
\end{align}
Which simplifies to:
\begin{equation}
    \bra{I} \hat{a}^\dagger_p \hat{a}_r \hat{a}^\dagger_r \hat{a}_s =   \bra{I} \hat{a}^\dagger_p \hat{a}_s
\end{equation}


\subsubsection{Summary for the Hamiltonian}
In short this we shall sum up what appears to be the minimal amount of operators required to retrieve all information for the Hamiltonian for the two-electron (same spin) operators.
The value:
\begin{equation}
  \frac{1}{2} (g_{pqrs} + g_{rspq} - g_{rqps} - g_{psrq})
\end{equation}
 Can be retrieved for any of the follwing non-vanishing operator sequence combinations yielding a higher and the same (thus non redundant) address for a given $\bra{I}$:
\begin{enumerate}
  \item $s > r > p > q$
  \item $s > r > q > p$
  \item $s > q > r > p$
\end{enumerate}
For every occupied index $x$ in an ONV:
\begin{equation}
  \frac{1}{2} (g_{xxpq} + g_{pqxx} - g_{xpqx})
\end{equation}
We can also see that for $p=x$ we arrive at:
\begin{equation}
  \frac{1}{2} (g_{xxxq})
\end{equation}
For every unnocupied index $y$ ($p \neq y$) in an ONV:
\begin{equation}
  \frac{1}{2} (g_{pyyq})
\end{equation}
Where for both cases $p<q$.
\subsection{Calculating Addresses}

\section{The FCI (restricted) matrix vector product}

Internally it is recommended to work with relative addresses for both spin functions and then order them logically in your total vectors. Given the nature of our Hamiltonian (operators do not affect spin) we can write an ONV as two ONVs from two seperate Fock spaces (and have the total Fock space be a direct product of those spaces). $\ket{ONV} = \ket{ONV_\alpha ONV_\beta}$. Choose one of the ONVs to be major and the other minor. If for example your $\alpha$ Fock space is major your $ONV_\alpha$ will permutate after a full permutation of your $ONV_\beta$ which are $\beta$ Fock space dimension permutations.
We define $\textbf{dim}_\alpha$ as the $\alpha$ Fock space dimension and $\textbf{dim}_\beta$ as the $\beta$ Fock space dimension and the total dimension $\textbf{dim}_{total} = \textbf{dim}_\alpha * \textbf{dim}_\beta$. \\
An eigenvector is thus stored as such:
\begin{tabular}{|l|l||l|}
\hline
$I_{\alpha}$ & $I_{\beta}$ & $I_{\textbf{total}}$ \\ \hline
$I_1$ & $I_1$ & $I_1$ \\ \hline
$I_1$ & $I_2$ & $I_2$ \\ \hline
$I_1$ & $I_3$ & $I_3$ \\ \hline
$I_1$ & $I_{...}$ & $I_{...}$ \\ \hline
$I_1$ & $I_{\textbf{dim}_\beta}$ & $I_{\textbf{dim}_\beta}$ \\ \hline
$I_2$ & $I_{1}$ & $I_{\textbf{dim}_\beta + 1}$ \\ \hline
$I_2$ & $I_{2}$ & $I_{\textbf{dim}_\beta + 2}$ \\ \hline
$I_{...}$ & $I_{...}$ & $I_{... * \textbf{dim}_\beta + ...}$ \\ \hline
$I_{\textbf{dim}_\alpha}$ & $I_{\textbf{dim}_\beta}$ & $I_{\textbf{dim}_{total}}$ \\ \hline
\end{tabular}
\\
Rudimentary approach to performing the matrix vector product for a none storable Hamiltonian can be done as follows (where P is the vector resulting from the product and X is the vector partaking in the product):
\begin{align}
  \text{P}_{I} = \sum_J \text{H}_{IJ} * \text{X}_{J} \\
  \text{P}_{J} = \sum_I \text{H}_{JI} * \text{X}_{I}
\end{align}
Where $\text{Hamiltonian}_{JI} = \text{Hamiltonian}_{IJ}$ (or $\text{Hamiltonian}_{JI} = \text{Hamiltonian}_{IJ}^*$ when working complex). Allowing for the upperdiagonal approach (for I coupling to J one finds J coupling to I)

\subsection{Evaluations Seperated by Spin}
For a set of operators only affecting one spin function we can repeat the coupling as many times as there are permutations in the Fock space of the opposite spin. e.g.:
\begin{equation} \label{eq:coupling_alpha}
  \bra{I_\alpha I_\beta} \hat{a}^{\dagger}_{p \alpha} \hat{a}_{q \alpha} \hat{a}^{\dagger}_{r \alpha} \hat{a}_{s \alpha} \ket{J_\alpha I_\beta} = 1
\end{equation}
Equation (\ref{eq:coupling_alpha}) holds for any $I_\beta$.
Allowing to re-use calculated Hamiltonian elements related to the coupling.
\begin{align}
  \forall I_\beta: \text{P}_{I_\alpha * \textbf{dim}_\beta + I_\beta} = \sum_{J_\alpha > I_\alpha} \text{H}^{\alpha}_{I_\alpha J_\alpha} * \text{X}_{J_\alpha * \textbf{dim}_\beta + I_\beta} \\
  \forall I_\beta: \text{P}_{J_\alpha * \textbf{dim}_\beta + I_\beta} = \sum_{I_\alpha > J_\alpha} \text{H}^{\alpha}_{J_\alpha I_\alpha} * \text{X}_{I_\alpha * \textbf{dim}_\beta + I_\beta}
\end{align}
Where the $\text{H}^{\alpha}$ refers to the Hamiltonian values retrievable by exclusively operators working on alpha electrons.

Base-line implementation (with a for loop) will result significant difference in execution speed for both spin functions.
As an example we have an FCI calculation with $K=12$ molecular orbitals and 6 $\alpha$ electrons and 6 $\beta$ electrons.
Yielding an Alpha and Beta Fock space of dimenion 924 and a total dimension of 853.776k where alpha is major as in the vector example above.
Calculating all couplings with Alpha and perfoming the matvec takes: $249 ms$ versus Beta: $2561 ms$.
The alpha matvec is a lot cheaper because it gets repeated in an uninterrupted sequencel matter, versus beta which repeats with a large interval (the dimenion of the alpha Fock space) causing cache misses. This was solved using the Eigen3 (ref EIGEN) API for efficiently performing the matvec in a vectorized matter. (There is no motivation other than educational to attempt to write a library with similar functionallity and performance as Eigen3 by yourself)

The resulting vector and the vector from the product can both be mapped to a matrix representation.

\begin{equation}
\begin{bmatrix} \label{eq:matrixrep}
    I^{\alpha}_1 I^{\beta}_1 & I^{\alpha}_1 I^{\beta}_2  & I^{\alpha}_1 I^{\beta}_{...} & \dots  & I^{\alpha}_1 I^{\beta}_{\textbf{dim}_\beta} \\
    I^{\alpha}_2 I^{\beta}_1 & I^{\alpha}_2 I^{\beta}_2  & I^{\alpha}_2 I^{\beta}_{...} & \dots  & I^{\alpha}_2 I^{\beta}_{\textbf{dim}_\beta} \\
    I^{\alpha}_{...} I^{\beta}_1 & I^{\alpha}_{...} I^{\beta}_2  & I^{\alpha}_{...} I^{\beta}_{...} & \dots  & I^{\alpha}_{...} I^{\beta}_{\textbf{dim}_\beta} \\
    \vdots & \vdots & \vdots & \ddots & \vdots \\
    I^{\alpha}_{\textbf{dim}_\alpha} I^{\beta}_1 & I^{\alpha}_{\textbf{dim}_\alpha} I^{\beta}_2  & I^{\alpha}_{\textbf{dim}_\alpha} I^{\beta}_{...} & \dots  &
    I^{\alpha}_{\textbf{dim}_\alpha} I^{\beta}_{\textbf{dim}_\beta}
\end{bmatrix}
\end{equation}

One can then calculate $\text{H}^{\alpha}$ and $\text{H}^{\beta}$ which are $\textbf{dim}_\alpha$ by $\textbf{dim}_\alpha$ and $\textbf{dim}_\beta$ by $\textbf{dim}_\beta$
matrixes respectively. Note that these will be sparse and will only cost a fraction to store compared P and X.

We can then calculate a portion of P (all one-electron and part of the two-electron (only the same spin) evaluations), $\text{P}^{\alpha}$ and $\text{P}^{\beta}$.
Where we now define $\text{P} = \text{P}^{\alpha \beta} + \text{P}^{\alpha} + \text{P}^{\beta}$ with $\text{P}^{\alpha \beta}$ exclusively operator combinations working on both spin functions simultaneously. Let us call $\mathcal{P}$: P mapped as the matrix in eqation (\ref{eq:matrixrep})

\begin{align}
    \forall I_\beta: \mathcal{P}_{I_\alpha I_\beta} & = \sum_{J_\alpha} \text{H}^{\alpha}_{I_\alpha J_\alpha} * \text{X}_{J_\alpha I_\beta} \\
    \forall I_\alpha: \mathcal{P}_{I_\alpha  I_\beta} & = \sum_{J_\beta} \text{H}^{\beta}_{I_\beta J_\beta} * \text{X}_{I_\alpha J_\beta} \\
   \mathcal{P}^\alpha & = \text{H}^{\alpha} * \textbf{X} \\
  \mathcal{P}^\beta & = \textbf{X} * \text{H}^{\beta}
\end{align}


\end{document}
